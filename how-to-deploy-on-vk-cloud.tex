% !TEX program = xelatex
\documentclass[14pt, a4paper]{extarticle}
\usepackage[english, russian]{babel}
\usepackage{amsmath}
\usepackage{mathastext}
\usepackage{xcolor}
\usepackage{accsupp}
\usepackage{listings}
\usepackage[unicode, colorlinks=true, linkcolor=blue]{hyperref}
\usepackage{geometry}
\usepackage{graphicx}
\usepackage{latexsym}
\usepackage{amssymb}
\usepackage{indentfirst}
\usepackage{tikz}
\usepackage{pgfplots}
\usepackage{bookmark} 
\usepackage{fontspec} 
\usepackage{titlesec}
\titleformat{\section}{\large\scshape\color{gray}}{\hspace*{-4.5em}\rule[-1.35mm]{4.5em}{1.25em}{\color{white}\hspace{-1cm}\large\scshape\thesection\hspace{5pt}}}{22pt}{}
% \titleformat{\section}{\large\scshape}{\thesection}{10pt}{}
\titleformat{\subsection}{\large\scshape\color{gray}}{\hspace*{-4.5em}\rule[-1.35mm]{4.5em}{1.25em}{\color{white}\hspace{-1cm}\large\scshape\thesubsection\hspace{5pt}}}{10pt}{}
% \titleformat{\subsection}{\large\scshape}{\thesubsection}{10pt}{}

\setmainfont{Times New Roman}
\linespread{1.5}

\pgfplotsset{compat=1.7}

\definecolor{codegreen}{rgb}{0,0.6,0}
\definecolor{codegray}{rgb}{0.5,0.5,0.5}
\definecolor{codepurple}{rgb}{0.58,0,0.82}
\definecolor{backcolour}{rgb}{0.95,0.95,0.92}
\newcommand{\noncopynumber}[1]{%
    \BeginAccSupp{method=escape,ActualText={}}%
    #1%
    \EndAccSupp{}%
}
\lstdefinestyle{mycodestyle}{
    frame=single,
    backgroundcolor=\color{backcolour},
    commentstyle=\color{codegreen},
    keywordstyle=\color{blue},
    numbers=left,
    numberstyle=\color{codegray},
    breaklines=true,
    basicstyle=\ttfamily,
    breakatwhitespace=false,
    captionpos=b,
    keepspaces=true,
    showspaces=false,
    showstringspaces=false,
    showtabs=false,
    columns=fullflexible,
    numberstyle=\noncopynumber,
    language=C++,
    alsolanguage=PHP
}
\lstset{style=mycodestyle}
% \begin{lstlisting}
%     #include <iostream>
    
%     int main() {
%         std::cout << "Hello, world!" << std::endl;
%         return 0;
%     }
% \end{lstlisting}

\geometry{left=30mm}
\geometry{right=10mm}
\geometry{top=20mm}
\geometry{bottom=20mm}

\graphicspath{{img/}}

\def\[#1\]{\begin{equation}\begin{aligned}#1\end{aligned}\end{equation}}

\begin{document}
\begin{titlepage}
    \centering
    \par
    {\scshape ITMO University\par}
    \vspace{80mm}
    {\Large\bfseries\scshape Как задеплоить сайт vite на vk cloud\par}
    \vspace{20mm}
    {\scshape Кербер Егор\par}
    \vfill
    {\scshape\today\par}
\end{titlepage}

\setcounter{page}{2}
\tableofcontents
\newpage

\section{Регистрация на vk cloud и добавление правила}

\begin{enumerate}
    \item Регисрируемся на vk cloud 
    \item Создаем инстанс 
    
    \includegraphics*[width=0.8\linewidth]{img/2024-02-17-17-41-09.png}

    \item Желательно выбираем Ubuntu
    
    \includegraphics*[width=0.8\linewidth]{img/2024-02-17-17-42-14.png}

    \item Делаем следующие шаги и все, инстанс создан 
    \item В разделе 'Виртуальные сети'-'Настройки firewall' добавляем новую настройку firewall 

    \includegraphics*[width=0.8\linewidth]{img/2024-02-17-17-57-58.png}    
    
    \item Обзываем web 

    \includegraphics*[width=0.8\linewidth]{img/2024-02-17-18-05-15.png}
    \item Делаем настройку 
    
    \includegraphics*[width=0.8\linewidth]{img/2024-02-17-18-06-39.png}

    \item Добавляем 2 новых правила
    
    \includegraphics*[width=0.8\linewidth]{img/2024-02-17-18-08-48.png}

    \includegraphics*[width=0.5\linewidth]{img/2024-02-17-19-18-22.png}

    \includegraphics*[width=0.5\linewidth]{img/2024-02-17-19-18-42.png}

    \item Добавляем правило к сети

    \includegraphics*[width=0.8\linewidth]{img/2024-02-17-19-20-42.png}

    \includegraphics*[width=0.8\linewidth]{img/2024-02-17-19-22-20.png}

    Это правило нужно для того, чтобы сайт вскоре был доступен 
    
    по http://212.111.84.6:8000/
\end{enumerate}

\newpage
\section{Настройка nginx}

\begin{enumerate}
    \item При помощи данной команды подключаемся по ssh к удаленной виртуальной машине
    
    \includegraphics*[width=0.8\linewidth]{img/2024-02-17-19-39-01.png}
    
    По следующему адресу кстати будет доступен сайт
    
    \includegraphics*[width=0.8\linewidth]{img/2024-02-17-19-54-30.png}
    
    Там как и в обычной линухе делаем sudo bash, чтобы удобнее было.
    
    Потом клонируем репозиторий, запускаем контейнер при помощи docker-compose up -d 
    
    (-d чтобы работа терминала не зависала только на этом процессе)
    
    далее по необходимости делаем docker exec -it vite\_docker sh
    
    чтобы войти в контейнер и выполнить команды npm i и потом npm run dev 
    
    Все, сервер поднялся, теперь конфигурируем сам nginx. 

    \item В папке conf.d делаем конфигурацию при помощи команды nano 
    
    /etc/nginx/conf.d/resume.conf 
    
    \includegraphics*[width=0.8\linewidth]{img/2024-02-17-20-27-15.png}

    \item Делаем более общую конфигурацию nano /etc/nginx/nginx.conf
    
    \begin{lstlisting}
user www-data;
worker_processes auto;
pid /run/nginx.pid;
include /etc/nginx/modules-enabled/*.conf;

events {
    worker_connections 768;
    # multi_accept on;
}

http {
    ##
    # Basic Settings
    ##

    sendfile on;
    tcp_nopush on;
    tcp_nodelay on;
    keepalive_timeout 65;
    types_hash_max_size 2048;

    include /etc/nginx/mime.types;
    default_type application/octet-stream;

    ##
    # SSL Settings
    ##

    #ssl_protocols TLSv1 TLSv1.1 TLSv1.2; # Dropping SSLv3, ref: POODLE
    #ssl_prefer_server_ciphers on;

    ##
    # Logging Settings
    ##

    access_log /var/log/nginx/access.log;
    error_log /var/log/nginx/error.log;

    ##
    # Gzip Settings
    ##

    gzip on;

    ##
    # Virtual Host Configs
    ##

    include /etc/nginx/conf.d/*.conf;
    include /etc/nginx/sites-enabled/*;
}
    \end{lstlisting}

    \item Выполняем команду systemctl restart nginx чтобы запустить nginx
\end{enumerate}

\newpage
\section{Покупка домена}

\begin{enumerate}
    \item Заходим на рег.ру, регистрируемся и т.п. и покупаем домен 
    
    \includegraphics*[width=0.8\linewidth]{img/2024-02-17-20-04-40.png}

    \item Потом заходим в личный кабинет, хостинги
    
    \includegraphics*[width=0.8\linewidth]{img/2024-02-17-20-06-31.png}

    \item Заходим в нужный хостинг, мотаем вниз, заходим в панель управления
    
    \includegraphics*[width=0.8\linewidth]{img/2024-02-17-20-07-22.png}

    \item Заходим в управление DNS 
    
    \includegraphics*[width=0.8\linewidth]{img/2024-02-17-20-08-22.png}

    \item Управлять DNS записями 
    
    \includegraphics*[width=0.8\linewidth]{img/2024-02-17-20-09-03.png}

    \item Создаем запись
    
    \includegraphics*[width=0.8\linewidth]{img/2024-02-17-20-09-44.png}

    \item Делаем так

    \includegraphics*[width=0.8\linewidth]{img/2024-02-17-20-10-30.png}
\end{enumerate}

\newpage
\section{Конфигурация vite и docker-compose}

Для vite важно добавить это поле 

\includegraphics*[width=0.8\linewidth]{img/2024-02-17-20-12-31.png}

В порте указываем внешний порт (watch usePolling добавил на всякий случай, нужен вроде чтобы hot reload работал или что-то такое)

Для docker-а делаем следующее

\includegraphics*[width=0.8\linewidth]{img/2024-02-17-20-14-32.png}

\subsection{Обьяснение конфигурации от ChatGPT}

Этот файл является файлом конфигурации Docker Compose, который используется для определения и настройки многоконтейнерных приложений. Давайте разберем каждую часть этого файла:

version: "3.4": Это указывает на версию формата файла Docker Compose, которую следует использовать при интерпретации этого файла. В данном случае используется версия 3.4.

services:: Этот ключевой раздел определяет список всех сервисов, которые будут запущены при помощи Docker Compose.

vite\_docker:: Это имя сервиса, которое мы определяем. Здесь vite\_docker - это произвольное имя, которое мы дали нашему сервису.

image: node:alpine: Это образ Docker, который будет использоваться для запуска нашего сервиса. Здесь мы используем образ node:alpine, основанный на Alpine Linux, который содержит Node.js.

container\_name: vite\_docker: Это имя контейнера, которое будет присвоено контейнеру, когда он будет запущен.

entrypoint: /bin/sh: Это команда, которая будет выполнена при запуске контейнера. Здесь мы указываем запуск интерактивной оболочки sh в контейнере.

ports:: Этот параметр определяет порты, которые будут проброшены из контейнера на хостовую машину. В данном случае, порт 8000 контейнера будет проброшен на порт 8000 хостовой машины.

working\_dir: /app: Это рабочий каталог внутри контейнера, к которому будет сделано изменение после запуска контейнера.

volumes:: Этот параметр позволяет примонтировать локальные файловые системы или тома внутрь контейнера. Здесь используется тип bind, что означает привязку (mount) каталога или файла с хостовой машины к каталогу в контейнере.

type: bind: Это тип тома, который мы используем для монтирования файловой системы хоста.

source: ./: Это путь к каталогу или файлу на хостовой машине, который будет примонтирован в контейнер.

target: /app: Это место в контейнере, куда будет примонтирован каталог или файл с хостовой машины.

tty: true: Этот параметр устанавливает взаимодействие с терминалом (TTY) внутри контейнера, что обеспечивает возможность использования интерактивной оболочки внутри контейнера.

Этот файл Docker Compose определяет контейнер, который будет запущен с образом Node.js, пробрасывает порт 8000, монтирует текущий рабочий каталог хостовой машины внутрь контейнера и запускает интерактивную оболочку sh внутри контейнера.

\section*{Ссылки}

\href{https://dev.to/ysmnikhil/how-to-build-with-react-or-vue-with-vite-and-docker-1a3l}{https://dev.to/ysmnikhil/how-to-build-with-react-or-vue-with-vite-and-docker-1a3l}

\end{document}